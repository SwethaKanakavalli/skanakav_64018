% Options for packages loaded elsewhere
\PassOptionsToPackage{unicode}{hyperref}
\PassOptionsToPackage{hyphens}{url}
%
\documentclass[
]{article}
\title{Assignment\_2}
\author{Swetha}
\date{9/20/2022}

\usepackage{amsmath,amssymb}
\usepackage{lmodern}
\usepackage{iftex}
\ifPDFTeX
  \usepackage[T1]{fontenc}
  \usepackage[utf8]{inputenc}
  \usepackage{textcomp} % provide euro and other symbols
\else % if luatex or xetex
  \usepackage{unicode-math}
  \defaultfontfeatures{Scale=MatchLowercase}
  \defaultfontfeatures[\rmfamily]{Ligatures=TeX,Scale=1}
\fi
% Use upquote if available, for straight quotes in verbatim environments
\IfFileExists{upquote.sty}{\usepackage{upquote}}{}
\IfFileExists{microtype.sty}{% use microtype if available
  \usepackage[]{microtype}
  \UseMicrotypeSet[protrusion]{basicmath} % disable protrusion for tt fonts
}{}
\makeatletter
\@ifundefined{KOMAClassName}{% if non-KOMA class
  \IfFileExists{parskip.sty}{%
    \usepackage{parskip}
  }{% else
    \setlength{\parindent}{0pt}
    \setlength{\parskip}{6pt plus 2pt minus 1pt}}
}{% if KOMA class
  \KOMAoptions{parskip=half}}
\makeatother
\usepackage{xcolor}
\IfFileExists{xurl.sty}{\usepackage{xurl}}{} % add URL line breaks if available
\IfFileExists{bookmark.sty}{\usepackage{bookmark}}{\usepackage{hyperref}}
\hypersetup{
  pdftitle={Assignment\_2},
  pdfauthor={Swetha},
  hidelinks,
  pdfcreator={LaTeX via pandoc}}
\urlstyle{same} % disable monospaced font for URLs
\usepackage[margin=1in]{geometry}
\usepackage{color}
\usepackage{fancyvrb}
\newcommand{\VerbBar}{|}
\newcommand{\VERB}{\Verb[commandchars=\\\{\}]}
\DefineVerbatimEnvironment{Highlighting}{Verbatim}{commandchars=\\\{\}}
% Add ',fontsize=\small' for more characters per line
\usepackage{framed}
\definecolor{shadecolor}{RGB}{248,248,248}
\newenvironment{Shaded}{\begin{snugshade}}{\end{snugshade}}
\newcommand{\AlertTok}[1]{\textcolor[rgb]{0.94,0.16,0.16}{#1}}
\newcommand{\AnnotationTok}[1]{\textcolor[rgb]{0.56,0.35,0.01}{\textbf{\textit{#1}}}}
\newcommand{\AttributeTok}[1]{\textcolor[rgb]{0.77,0.63,0.00}{#1}}
\newcommand{\BaseNTok}[1]{\textcolor[rgb]{0.00,0.00,0.81}{#1}}
\newcommand{\BuiltInTok}[1]{#1}
\newcommand{\CharTok}[1]{\textcolor[rgb]{0.31,0.60,0.02}{#1}}
\newcommand{\CommentTok}[1]{\textcolor[rgb]{0.56,0.35,0.01}{\textit{#1}}}
\newcommand{\CommentVarTok}[1]{\textcolor[rgb]{0.56,0.35,0.01}{\textbf{\textit{#1}}}}
\newcommand{\ConstantTok}[1]{\textcolor[rgb]{0.00,0.00,0.00}{#1}}
\newcommand{\ControlFlowTok}[1]{\textcolor[rgb]{0.13,0.29,0.53}{\textbf{#1}}}
\newcommand{\DataTypeTok}[1]{\textcolor[rgb]{0.13,0.29,0.53}{#1}}
\newcommand{\DecValTok}[1]{\textcolor[rgb]{0.00,0.00,0.81}{#1}}
\newcommand{\DocumentationTok}[1]{\textcolor[rgb]{0.56,0.35,0.01}{\textbf{\textit{#1}}}}
\newcommand{\ErrorTok}[1]{\textcolor[rgb]{0.64,0.00,0.00}{\textbf{#1}}}
\newcommand{\ExtensionTok}[1]{#1}
\newcommand{\FloatTok}[1]{\textcolor[rgb]{0.00,0.00,0.81}{#1}}
\newcommand{\FunctionTok}[1]{\textcolor[rgb]{0.00,0.00,0.00}{#1}}
\newcommand{\ImportTok}[1]{#1}
\newcommand{\InformationTok}[1]{\textcolor[rgb]{0.56,0.35,0.01}{\textbf{\textit{#1}}}}
\newcommand{\KeywordTok}[1]{\textcolor[rgb]{0.13,0.29,0.53}{\textbf{#1}}}
\newcommand{\NormalTok}[1]{#1}
\newcommand{\OperatorTok}[1]{\textcolor[rgb]{0.81,0.36,0.00}{\textbf{#1}}}
\newcommand{\OtherTok}[1]{\textcolor[rgb]{0.56,0.35,0.01}{#1}}
\newcommand{\PreprocessorTok}[1]{\textcolor[rgb]{0.56,0.35,0.01}{\textit{#1}}}
\newcommand{\RegionMarkerTok}[1]{#1}
\newcommand{\SpecialCharTok}[1]{\textcolor[rgb]{0.00,0.00,0.00}{#1}}
\newcommand{\SpecialStringTok}[1]{\textcolor[rgb]{0.31,0.60,0.02}{#1}}
\newcommand{\StringTok}[1]{\textcolor[rgb]{0.31,0.60,0.02}{#1}}
\newcommand{\VariableTok}[1]{\textcolor[rgb]{0.00,0.00,0.00}{#1}}
\newcommand{\VerbatimStringTok}[1]{\textcolor[rgb]{0.31,0.60,0.02}{#1}}
\newcommand{\WarningTok}[1]{\textcolor[rgb]{0.56,0.35,0.01}{\textbf{\textit{#1}}}}
\usepackage{graphicx}
\makeatletter
\def\maxwidth{\ifdim\Gin@nat@width>\linewidth\linewidth\else\Gin@nat@width\fi}
\def\maxheight{\ifdim\Gin@nat@height>\textheight\textheight\else\Gin@nat@height\fi}
\makeatother
% Scale images if necessary, so that they will not overflow the page
% margins by default, and it is still possible to overwrite the defaults
% using explicit options in \includegraphics[width, height, ...]{}
\setkeys{Gin}{width=\maxwidth,height=\maxheight,keepaspectratio}
% Set default figure placement to htbp
\makeatletter
\def\fps@figure{htbp}
\makeatother
\setlength{\emergencystretch}{3em} % prevent overfull lines
\providecommand{\tightlist}{%
  \setlength{\itemsep}{0pt}\setlength{\parskip}{0pt}}
\setcounter{secnumdepth}{-\maxdimen} % remove section numbering
\ifLuaTeX
  \usepackage{selnolig}  % disable illegal ligatures
\fi

\begin{document}
\maketitle

\begin{Shaded}
\begin{Highlighting}[]
\CommentTok{\# Loading the "lpSolve","lpSolveAPI" packages.}
\FunctionTok{library}\NormalTok{(lpSolve)}
\end{Highlighting}
\end{Shaded}

\begin{verbatim}
## Warning: package 'lpSolve' was built under R version 4.1.3
\end{verbatim}

\begin{Shaded}
\begin{Highlighting}[]
\FunctionTok{library}\NormalTok{(lpSolveAPI)}
\end{Highlighting}
\end{Shaded}

\begin{verbatim}
## Warning: package 'lpSolveAPI' was built under R version 4.1.3
\end{verbatim}

\begin{Shaded}
\begin{Highlighting}[]
\CommentTok{\# setting up the working directory}
\FunctionTok{getwd}\NormalTok{()}
\end{Highlighting}
\end{Shaded}

\begin{verbatim}
## [1] "C:/Users/mercy/OneDrive/Desktop/QMM/Assignment_2"
\end{verbatim}

\begin{Shaded}
\begin{Highlighting}[]
\FunctionTok{setwd}\NormalTok{(}\StringTok{"C:/Users/mercy/OneDrive/Desktop/QMM/Assignment\_2"}\NormalTok{)}
\end{Highlighting}
\end{Shaded}

\hypertarget{from-the-lp-problem-we-have-to-find-objective-function-constraints-and-decision-variables.}{%
\section{From the LP problem we have to find objective function,
constraints and decision
variables.}\label{from-the-lp-problem-we-have-to-find-objective-function-constraints-and-decision-variables.}}

\begin{Shaded}
\begin{Highlighting}[]
\CommentTok{\# starting the lp problem with 12 constraints and 9 decision variables.}
\NormalTok{lprec }\OtherTok{\textless{}{-}} \FunctionTok{make.lp}\NormalTok{(}\DecValTok{12}\NormalTok{,}\DecValTok{9}\NormalTok{)}
\end{Highlighting}
\end{Shaded}

\begin{Shaded}
\begin{Highlighting}[]
\CommentTok{\# Setting the objective function for the problem.}
\FunctionTok{set.objfn}\NormalTok{(lprec, }\FunctionTok{c}\NormalTok{(}\DecValTok{420}\NormalTok{,}\DecValTok{420}\NormalTok{,}\DecValTok{420}\NormalTok{,}\DecValTok{360}\NormalTok{,}\DecValTok{360}\NormalTok{,}\DecValTok{360}\NormalTok{,}\DecValTok{300}\NormalTok{,}\DecValTok{300}\NormalTok{,}\DecValTok{300}\NormalTok{))}
\CommentTok{\# and Changing the direction to setup maximization}
\FunctionTok{lp.control}\NormalTok{(lprec, }\AttributeTok{sense =} \StringTok{"max"}\NormalTok{)}
\end{Highlighting}
\end{Shaded}

\begin{verbatim}
## $anti.degen
## [1] "fixedvars" "stalling" 
## 
## $basis.crash
## [1] "none"
## 
## $bb.depthlimit
## [1] -50
## 
## $bb.floorfirst
## [1] "automatic"
## 
## $bb.rule
## [1] "pseudononint" "greedy"       "dynamic"      "rcostfixing" 
## 
## $break.at.first
## [1] FALSE
## 
## $break.at.value
## [1] 1e+30
## 
## $epsilon
##       epsb       epsd      epsel     epsint epsperturb   epspivot 
##      1e-10      1e-09      1e-12      1e-07      1e-05      2e-07 
## 
## $improve
## [1] "dualfeas" "thetagap"
## 
## $infinite
## [1] 1e+30
## 
## $maxpivot
## [1] 250
## 
## $mip.gap
## absolute relative 
##    1e-11    1e-11 
## 
## $negrange
## [1] -1e+06
## 
## $obj.in.basis
## [1] TRUE
## 
## $pivoting
## [1] "devex"    "adaptive"
## 
## $presolve
## [1] "none"
## 
## $scalelimit
## [1] 5
## 
## $scaling
## [1] "geometric"   "equilibrate" "integers"   
## 
## $sense
## [1] "maximize"
## 
## $simplextype
## [1] "dual"   "primal"
## 
## $timeout
## [1] 0
## 
## $verbose
## [1] "neutral"
\end{verbatim}

\begin{Shaded}
\begin{Highlighting}[]
\CommentTok{\# Setting up all the constraint values row by row}
\CommentTok{\# Capacity constraints:}
\FunctionTok{set.row}\NormalTok{(lprec, }\DecValTok{1}\NormalTok{, }\FunctionTok{c}\NormalTok{(}\DecValTok{1}\NormalTok{,}\DecValTok{1}\NormalTok{,}\DecValTok{1}\NormalTok{), }\AttributeTok{indices =} \FunctionTok{c}\NormalTok{(}\DecValTok{1}\NormalTok{,}\DecValTok{4}\NormalTok{,}\DecValTok{7}\NormalTok{))}
\FunctionTok{set.row}\NormalTok{(lprec, }\DecValTok{2}\NormalTok{, }\FunctionTok{c}\NormalTok{(}\DecValTok{1}\NormalTok{,}\DecValTok{1}\NormalTok{,}\DecValTok{1}\NormalTok{), }\AttributeTok{indices =} \FunctionTok{c}\NormalTok{(}\DecValTok{2}\NormalTok{,}\DecValTok{5}\NormalTok{,}\DecValTok{8}\NormalTok{))}
\FunctionTok{set.row}\NormalTok{(lprec, }\DecValTok{3}\NormalTok{, }\FunctionTok{c}\NormalTok{(}\DecValTok{1}\NormalTok{,}\DecValTok{1}\NormalTok{,}\DecValTok{1}\NormalTok{), }\AttributeTok{indices =} \FunctionTok{c}\NormalTok{(}\DecValTok{3}\NormalTok{,}\DecValTok{6}\NormalTok{,}\DecValTok{9}\NormalTok{))}
\CommentTok{\# Storage constraints:}
\FunctionTok{set.row}\NormalTok{(lprec, }\DecValTok{4}\NormalTok{, }\FunctionTok{c}\NormalTok{(}\DecValTok{20}\NormalTok{,}\DecValTok{15}\NormalTok{,}\DecValTok{12}\NormalTok{), }\AttributeTok{indices =} \FunctionTok{c}\NormalTok{(}\DecValTok{1}\NormalTok{,}\DecValTok{4}\NormalTok{,}\DecValTok{7}\NormalTok{))}
\FunctionTok{set.row}\NormalTok{(lprec, }\DecValTok{5}\NormalTok{, }\FunctionTok{c}\NormalTok{(}\DecValTok{20}\NormalTok{,}\DecValTok{15}\NormalTok{,}\DecValTok{12}\NormalTok{), }\AttributeTok{indices =} \FunctionTok{c}\NormalTok{(}\DecValTok{2}\NormalTok{,}\DecValTok{5}\NormalTok{,}\DecValTok{8}\NormalTok{))}
\FunctionTok{set.row}\NormalTok{(lprec, }\DecValTok{6}\NormalTok{, }\FunctionTok{c}\NormalTok{(}\DecValTok{20}\NormalTok{,}\DecValTok{15}\NormalTok{,}\DecValTok{12}\NormalTok{), }\AttributeTok{indices =} \FunctionTok{c}\NormalTok{(}\DecValTok{3}\NormalTok{,}\DecValTok{6}\NormalTok{,}\DecValTok{9}\NormalTok{))}
\CommentTok{\# Sales constraints:}
\FunctionTok{set.row}\NormalTok{(lprec, }\DecValTok{7}\NormalTok{, }\FunctionTok{c}\NormalTok{(}\DecValTok{1}\NormalTok{,}\DecValTok{1}\NormalTok{,}\DecValTok{1}\NormalTok{), }\AttributeTok{indices =} \FunctionTok{c}\NormalTok{(}\DecValTok{1}\NormalTok{,}\DecValTok{2}\NormalTok{,}\DecValTok{3}\NormalTok{))}
\FunctionTok{set.row}\NormalTok{(lprec, }\DecValTok{8}\NormalTok{, }\FunctionTok{c}\NormalTok{(}\DecValTok{1}\NormalTok{,}\DecValTok{1}\NormalTok{,}\DecValTok{1}\NormalTok{), }\AttributeTok{indices =} \FunctionTok{c}\NormalTok{(}\DecValTok{4}\NormalTok{,}\DecValTok{5}\NormalTok{,}\DecValTok{6}\NormalTok{))}
\FunctionTok{set.row}\NormalTok{(lprec, }\DecValTok{9}\NormalTok{, }\FunctionTok{c}\NormalTok{(}\DecValTok{1}\NormalTok{,}\DecValTok{1}\NormalTok{,}\DecValTok{1}\NormalTok{), }\AttributeTok{indices =} \FunctionTok{c}\NormalTok{(}\DecValTok{7}\NormalTok{,}\DecValTok{8}\NormalTok{,}\DecValTok{9}\NormalTok{))}
\CommentTok{\# Capacity usage constaints:}
\FunctionTok{set.row}\NormalTok{(lprec, }\DecValTok{10}\NormalTok{, }\FunctionTok{c}\NormalTok{(}\DecValTok{900}\NormalTok{,}\DecValTok{900}\NormalTok{,}\DecValTok{900}\NormalTok{,}\SpecialCharTok{{-}}\DecValTok{750}\NormalTok{,}\SpecialCharTok{{-}}\DecValTok{750}\NormalTok{,}\SpecialCharTok{{-}}\DecValTok{750}\NormalTok{), }\AttributeTok{indices =} \FunctionTok{c}\NormalTok{(}\DecValTok{1}\NormalTok{,}\DecValTok{4}\NormalTok{,}\DecValTok{7}\NormalTok{,}\DecValTok{2}\NormalTok{,}\DecValTok{5}\NormalTok{,}\DecValTok{8}\NormalTok{))}
\FunctionTok{set.row}\NormalTok{(lprec, }\DecValTok{11}\NormalTok{, }\FunctionTok{c}\NormalTok{(}\DecValTok{450}\NormalTok{,}\DecValTok{450}\NormalTok{,}\DecValTok{450}\NormalTok{,}\SpecialCharTok{{-}}\DecValTok{900}\NormalTok{,}\SpecialCharTok{{-}}\DecValTok{900}\NormalTok{,}\SpecialCharTok{{-}}\DecValTok{900}\NormalTok{), }\AttributeTok{indices =} \FunctionTok{c}\NormalTok{(}\DecValTok{2}\NormalTok{,}\DecValTok{5}\NormalTok{,}\DecValTok{8}\NormalTok{,}\DecValTok{3}\NormalTok{,}\DecValTok{6}\NormalTok{,}\DecValTok{9}\NormalTok{))}
\FunctionTok{set.row}\NormalTok{(lprec, }\DecValTok{12}\NormalTok{, }\FunctionTok{c}\NormalTok{(}\DecValTok{450}\NormalTok{,}\DecValTok{450}\NormalTok{,}\DecValTok{450}\NormalTok{,}\SpecialCharTok{{-}}\DecValTok{750}\NormalTok{,}\SpecialCharTok{{-}}\DecValTok{750}\NormalTok{,}\SpecialCharTok{{-}}\DecValTok{750}\NormalTok{), }\AttributeTok{indices =} \FunctionTok{c}\NormalTok{(}\DecValTok{1}\NormalTok{,}\DecValTok{4}\NormalTok{,}\DecValTok{7}\NormalTok{,}\DecValTok{3}\NormalTok{,}\DecValTok{6}\NormalTok{,}\DecValTok{9}\NormalTok{))}
\end{Highlighting}
\end{Shaded}

\begin{Shaded}
\begin{Highlighting}[]
\CommentTok{\# Setting all the right hand side values}
\NormalTok{rhs }\OtherTok{\textless{}{-}} \FunctionTok{c}\NormalTok{(}\DecValTok{750}\NormalTok{,}\DecValTok{900}\NormalTok{,}\DecValTok{450}\NormalTok{,}\DecValTok{13000}\NormalTok{,}\DecValTok{12000}\NormalTok{,}\DecValTok{5000}\NormalTok{,}\DecValTok{900}\NormalTok{,}\DecValTok{1200}\NormalTok{,}\DecValTok{750}\NormalTok{,}\DecValTok{0}\NormalTok{,}\DecValTok{0}\NormalTok{,}\DecValTok{0}\NormalTok{)}
\FunctionTok{set.rhs}\NormalTok{(lprec, rhs)}
\end{Highlighting}
\end{Shaded}

\begin{Shaded}
\begin{Highlighting}[]
\CommentTok{\# Set up the constraint type}
\FunctionTok{set.constr.type}\NormalTok{(lprec, }\FunctionTok{c}\NormalTok{(}\StringTok{"\textless{}="}\NormalTok{,}\StringTok{"\textless{}="}\NormalTok{,}\StringTok{"\textless{}="}\NormalTok{,}\StringTok{"\textless{}="}\NormalTok{,}\StringTok{"\textless{}="}\NormalTok{,}\StringTok{"\textless{}="}\NormalTok{,}\StringTok{"\textless{}="}\NormalTok{,}\StringTok{"\textless{}="}\NormalTok{,}\StringTok{"\textless{}="}\NormalTok{,}\StringTok{"="}\NormalTok{,}\StringTok{"="}\NormalTok{,}\StringTok{"="}\NormalTok{))}
\end{Highlighting}
\end{Shaded}

Now,For this problem all the values must be greater than 0.

\begin{Shaded}
\begin{Highlighting}[]
\CommentTok{\# Setting the boundary condiiton for the decision variables}
\FunctionTok{set.bounds}\NormalTok{(lprec, }\AttributeTok{lower =} \FunctionTok{rep}\NormalTok{(}\DecValTok{0}\NormalTok{, }\DecValTok{9}\NormalTok{))}
\end{Highlighting}
\end{Shaded}

\begin{Shaded}
\begin{Highlighting}[]
\CommentTok{\# Setting the names of the rows as constraints and the columns as decision variables.}
\NormalTok{lp.rownames }\OtherTok{\textless{}{-}} \FunctionTok{c}\NormalTok{(}\StringTok{"Plant 1 Capacity"}\NormalTok{, }\StringTok{"Plant 2 Capacity"}\NormalTok{, }\StringTok{"Plant 3 Capacity"}\NormalTok{, }\StringTok{"Plant 1 Storage"}\NormalTok{, }\StringTok{"Plant 2 Storage"}\NormalTok{, }\StringTok{"Plant 3 Storage"}\NormalTok{, }\StringTok{"Large Sales"}\NormalTok{, }\StringTok{"Medium Sales"}\NormalTok{, }\StringTok{"Small Sales"}\NormalTok{, }\StringTok{"Plant 1 and 2 Usage"}\NormalTok{, }\StringTok{"Plant 2 and 3 Usage"}\NormalTok{, }\StringTok{"Plant 1 and 3 Usage"}\NormalTok{)}
\NormalTok{lp.colnames }\OtherTok{\textless{}{-}} \FunctionTok{c}\NormalTok{(}\StringTok{"Plant 1L"}\NormalTok{, }\StringTok{"Plant 2L"}\NormalTok{, }\StringTok{"Plant 3L"}\NormalTok{, }\StringTok{"Plant 1M"}\NormalTok{, }\StringTok{"Plant 2M"}\NormalTok{, }\StringTok{"Plant 3M"}\NormalTok{, }\StringTok{"Plant 1S"}\NormalTok{, }\StringTok{"Plant 2S"}\NormalTok{, }\StringTok{"Plant 3S"}\NormalTok{)}
\FunctionTok{dimnames}\NormalTok{(lprec) }\OtherTok{\textless{}{-}} \FunctionTok{list}\NormalTok{(lp.rownames, lp.colnames)}
\end{Highlighting}
\end{Shaded}

Before running the code, this command should return the linear program
outline so we can verify that all the values are valid.

\begin{Shaded}
\begin{Highlighting}[]
\CommentTok{\# Now Return the linear programming object to ensure the values are correct}
\NormalTok{lprec}
\end{Highlighting}
\end{Shaded}

\begin{verbatim}
## Model name: 
##   a linear program with 9 decision variables and 12 constraints
\end{verbatim}

\begin{Shaded}
\begin{Highlighting}[]
\CommentTok{\# This model can also be saved to a file}
\FunctionTok{write.lp}\NormalTok{(lprec, }\AttributeTok{filename =} \StringTok{"Assignment{-}2.lp"}\NormalTok{, }\AttributeTok{type =} \StringTok{"lp"}\NormalTok{)}
\end{Highlighting}
\end{Shaded}

This following code will now look for an optimal solution. If it returns
``0'' value then that means the model has found an optimal solution.

\begin{Shaded}
\begin{Highlighting}[]
\CommentTok{\# Solving the linear program}
\FunctionTok{solve}\NormalTok{(lprec)}
\end{Highlighting}
\end{Shaded}

\begin{verbatim}
## [1] 0
\end{verbatim}

This model returned a ``0'', so it has found an optimal solution to the
problem.The function below will return what the maximum value for the
objective function will be.

\begin{Shaded}
\begin{Highlighting}[]
\CommentTok{\# Reviewing the objective function value}
\FunctionTok{get.objective}\NormalTok{(lprec)}
\end{Highlighting}
\end{Shaded}

\begin{verbatim}
## [1] 696000
\end{verbatim}

In this case, the maximum profits that can be achieved with these
constraints is \$696,000 per day.In order to determine how many units of
each kind of product the plants should produce, we will then return the
values of the decision variables.

\begin{Shaded}
\begin{Highlighting}[]
\CommentTok{\# Get the optimum decision variable values}
\FunctionTok{get.variables}\NormalTok{(lprec)}
\end{Highlighting}
\end{Shaded}

\begin{verbatim}
## [1] 516.6667   0.0000   0.0000 177.7778 666.6667   0.0000   0.0000 166.6667
## [9] 416.6667
\end{verbatim}

The Optimum decision variable values from the model:

Plant 1, Large: 516.67 units/day Plant 2, Large: 0 units/day Plant 3,
Large: 0 units/day Plant 1, Medium: 177.78 units/day Plant 2, Medium:
666.67 units/day Plant 3, Medium: 0 units/day Plant 1, Small: 0
units/day Plant 2, Small: 166.67 units/day Plant 3, Small: 416.67
units/day

The following two segments of code will tell us where our values fall
within the constraints, as well as return the surplus between the
constraint and the actual value from the constraints.

\begin{Shaded}
\begin{Highlighting}[]
\CommentTok{\# Get the constraint values for the problem}
\FunctionTok{get.constraints}\NormalTok{(lprec)}
\end{Highlighting}
\end{Shaded}

\begin{verbatim}
##  [1]   694.4444   833.3333   416.6667 13000.0000 12000.0000  5000.0000
##  [7]   516.6667   844.4444   583.3333     0.0000     0.0000     0.0000
\end{verbatim}

\begin{Shaded}
\begin{Highlighting}[]
\CommentTok{\# Reviewing the surplus for each constraint}
\FunctionTok{get.constraints}\NormalTok{(lprec) }\SpecialCharTok{{-}}\NormalTok{ rhs}
\end{Highlighting}
\end{Shaded}

\begin{verbatim}
##  [1] -5.555556e+01 -6.666667e+01 -3.333333e+01  0.000000e+00  0.000000e+00
##  [6] -9.094947e-13 -3.833333e+02 -3.555556e+02 -1.666667e+02  0.000000e+00
## [11]  0.000000e+00  0.000000e+00
\end{verbatim}

The ``get.sensitivity.rhs()'' function will be used to identify the
shadow prices for the right hand side and objective function of the
formulated linear programming problem.

\begin{Shaded}
\begin{Highlighting}[]
\FunctionTok{get.sensitivity.rhs}\NormalTok{(lprec)}
\end{Highlighting}
\end{Shaded}

\begin{verbatim}
## $duals
##  [1]    0.00    0.00    0.00   12.00   20.00   60.00    0.00    0.00    0.00
## [10]   -0.08    0.00    0.56    0.00  -40.00 -360.00    0.00    0.00 -120.00
## [19]  -24.00    0.00    0.00
## 
## $dualsfrom
##  [1] -1.000000e+30 -1.000000e+30 -1.000000e+30  1.122222e+04  1.150000e+04
##  [6]  4.800000e+03 -1.000000e+30 -1.000000e+30 -1.000000e+30  0.000000e+00
## [11] -1.000000e+30  0.000000e+00 -1.000000e+30 -1.000000e+02 -2.000000e+01
## [16] -1.000000e+30 -1.000000e+30 -4.444444e+01 -2.222222e+02 -1.000000e+30
## [21] -1.000000e+30
## 
## $dualstill
##  [1] 1.000000e+30 1.000000e+30 1.000000e+30 1.388889e+04 1.250000e+04
##  [6] 5.181818e+03 1.000000e+30 1.000000e+30 1.000000e+30 0.000000e+00
## [11] 1.000000e+30 0.000000e+00 1.000000e+30 1.000000e+02 2.500000e+01
## [16] 1.000000e+30 1.000000e+30 6.666667e+01 1.111111e+02 1.000000e+30
## [21] 1.000000e+30
\end{verbatim}

\begin{Shaded}
\begin{Highlighting}[]
\FunctionTok{get.sensitivity.obj}\NormalTok{(lprec) }
\end{Highlighting}
\end{Shaded}

\begin{verbatim}
## $objfrom
## [1]  3.60e+02 -1.00e+30 -1.00e+30  3.45e+02  3.45e+02 -1.00e+30 -1.00e+30
## [8]  2.52e+02  2.04e+02
## 
## $objtill
## [1] 4.60e+02 4.60e+02 7.80e+02 4.20e+02 4.20e+02 4.80e+02 3.24e+02 3.24e+02
## [9] 1.00e+30
\end{verbatim}

\hypertarget{dual-problem}{%
\subsection{Dual Problem:}\label{dual-problem}}

\begin{Shaded}
\begin{Highlighting}[]
\CommentTok{\# Begining the dual lp problem with 9 constraints and 12 decision variables.}
\NormalTok{lprecdual }\OtherTok{\textless{}{-}} \FunctionTok{make.lp}\NormalTok{(}\DecValTok{9}\NormalTok{,}\DecValTok{12}\NormalTok{)}
\end{Highlighting}
\end{Shaded}

\begin{Shaded}
\begin{Highlighting}[]
\CommentTok{\# Setting up the objective function for the problem.}
\FunctionTok{set.objfn}\NormalTok{(lprecdual, }\FunctionTok{c}\NormalTok{(}\DecValTok{750}\NormalTok{,}\DecValTok{900}\NormalTok{,}\DecValTok{450}\NormalTok{,}\DecValTok{13000}\NormalTok{,}\DecValTok{12000}\NormalTok{,}\DecValTok{5000}\NormalTok{,}\DecValTok{900}\NormalTok{,}\DecValTok{1200}\NormalTok{,}\DecValTok{750}\NormalTok{,}\DecValTok{0}\NormalTok{,}\DecValTok{0}\NormalTok{,}\DecValTok{0}\NormalTok{))}
\CommentTok{\# Changing the direction to set minimization}
\FunctionTok{lp.control}\NormalTok{(lprecdual, }\AttributeTok{sense =} \StringTok{"min"}\NormalTok{)}
\end{Highlighting}
\end{Shaded}

\begin{verbatim}
## $anti.degen
## [1] "fixedvars" "stalling" 
## 
## $basis.crash
## [1] "none"
## 
## $bb.depthlimit
## [1] -50
## 
## $bb.floorfirst
## [1] "automatic"
## 
## $bb.rule
## [1] "pseudononint" "greedy"       "dynamic"      "rcostfixing" 
## 
## $break.at.first
## [1] FALSE
## 
## $break.at.value
## [1] -1e+30
## 
## $epsilon
##       epsb       epsd      epsel     epsint epsperturb   epspivot 
##      1e-10      1e-09      1e-12      1e-07      1e-05      2e-07 
## 
## $improve
## [1] "dualfeas" "thetagap"
## 
## $infinite
## [1] 1e+30
## 
## $maxpivot
## [1] 250
## 
## $mip.gap
## absolute relative 
##    1e-11    1e-11 
## 
## $negrange
## [1] -1e+06
## 
## $obj.in.basis
## [1] TRUE
## 
## $pivoting
## [1] "devex"    "adaptive"
## 
## $presolve
## [1] "none"
## 
## $scalelimit
## [1] 5
## 
## $scaling
## [1] "geometric"   "equilibrate" "integers"   
## 
## $sense
## [1] "minimize"
## 
## $simplextype
## [1] "dual"   "primal"
## 
## $timeout
## [1] 0
## 
## $verbose
## [1] "neutral"
\end{verbatim}

\begin{Shaded}
\begin{Highlighting}[]
\CommentTok{\# Setting up all the constraint values row by row}
\FunctionTok{set.row}\NormalTok{(lprecdual, }\DecValTok{1}\NormalTok{, }\FunctionTok{c}\NormalTok{(}\DecValTok{1}\NormalTok{,}\DecValTok{20}\NormalTok{,}\DecValTok{1}\NormalTok{,}\DecValTok{900}\NormalTok{,}\DecValTok{450}\NormalTok{), }\AttributeTok{indices =} \FunctionTok{c}\NormalTok{(}\DecValTok{1}\NormalTok{,}\DecValTok{4}\NormalTok{,}\DecValTok{7}\NormalTok{,}\DecValTok{10}\NormalTok{,}\DecValTok{12}\NormalTok{))}
\FunctionTok{set.row}\NormalTok{(lprecdual, }\DecValTok{2}\NormalTok{, }\FunctionTok{c}\NormalTok{(}\DecValTok{1}\NormalTok{,}\DecValTok{15}\NormalTok{,}\DecValTok{1}\NormalTok{,}\DecValTok{900}\NormalTok{,}\DecValTok{450}\NormalTok{), }\AttributeTok{indices =} \FunctionTok{c}\NormalTok{(}\DecValTok{1}\NormalTok{,}\DecValTok{4}\NormalTok{,}\DecValTok{8}\NormalTok{,}\DecValTok{10}\NormalTok{,}\DecValTok{12}\NormalTok{))}
\FunctionTok{set.row}\NormalTok{(lprecdual, }\DecValTok{3}\NormalTok{, }\FunctionTok{c}\NormalTok{(}\DecValTok{1}\NormalTok{,}\DecValTok{12}\NormalTok{,}\DecValTok{1}\NormalTok{,}\DecValTok{900}\NormalTok{,}\DecValTok{450}\NormalTok{), }\AttributeTok{indices =} \FunctionTok{c}\NormalTok{(}\DecValTok{1}\NormalTok{,}\DecValTok{4}\NormalTok{,}\DecValTok{9}\NormalTok{,}\DecValTok{10}\NormalTok{,}\DecValTok{12}\NormalTok{))}
\FunctionTok{set.row}\NormalTok{(lprecdual, }\DecValTok{4}\NormalTok{, }\FunctionTok{c}\NormalTok{(}\DecValTok{1}\NormalTok{,}\DecValTok{20}\NormalTok{,}\DecValTok{1}\NormalTok{,}\SpecialCharTok{{-}}\DecValTok{750}\NormalTok{,}\DecValTok{450}\NormalTok{), }\AttributeTok{indices =} \FunctionTok{c}\NormalTok{(}\DecValTok{2}\NormalTok{,}\DecValTok{5}\NormalTok{,}\DecValTok{7}\NormalTok{,}\DecValTok{10}\NormalTok{,}\DecValTok{11}\NormalTok{))}
\FunctionTok{set.row}\NormalTok{(lprecdual, }\DecValTok{5}\NormalTok{, }\FunctionTok{c}\NormalTok{(}\DecValTok{1}\NormalTok{,}\DecValTok{15}\NormalTok{,}\DecValTok{1}\NormalTok{,}\SpecialCharTok{{-}}\DecValTok{750}\NormalTok{,}\DecValTok{450}\NormalTok{), }\AttributeTok{indices =} \FunctionTok{c}\NormalTok{(}\DecValTok{2}\NormalTok{,}\DecValTok{5}\NormalTok{,}\DecValTok{8}\NormalTok{,}\DecValTok{10}\NormalTok{,}\DecValTok{11}\NormalTok{))}
\FunctionTok{set.row}\NormalTok{(lprecdual, }\DecValTok{6}\NormalTok{, }\FunctionTok{c}\NormalTok{(}\DecValTok{1}\NormalTok{,}\DecValTok{12}\NormalTok{,}\DecValTok{1}\NormalTok{,}\SpecialCharTok{{-}}\DecValTok{750}\NormalTok{,}\DecValTok{450}\NormalTok{), }\AttributeTok{indices =} \FunctionTok{c}\NormalTok{(}\DecValTok{2}\NormalTok{,}\DecValTok{5}\NormalTok{,}\DecValTok{9}\NormalTok{,}\DecValTok{10}\NormalTok{,}\DecValTok{11}\NormalTok{))}
\FunctionTok{set.row}\NormalTok{(lprecdual, }\DecValTok{7}\NormalTok{, }\FunctionTok{c}\NormalTok{(}\DecValTok{1}\NormalTok{,}\DecValTok{20}\NormalTok{,}\DecValTok{1}\NormalTok{,}\SpecialCharTok{{-}}\DecValTok{900}\NormalTok{,}\SpecialCharTok{{-}}\DecValTok{450}\NormalTok{), }\AttributeTok{indices =} \FunctionTok{c}\NormalTok{(}\DecValTok{3}\NormalTok{,}\DecValTok{6}\NormalTok{,}\DecValTok{7}\NormalTok{,}\DecValTok{11}\NormalTok{,}\DecValTok{12}\NormalTok{))}
\FunctionTok{set.row}\NormalTok{(lprecdual, }\DecValTok{8}\NormalTok{, }\FunctionTok{c}\NormalTok{(}\DecValTok{1}\NormalTok{,}\DecValTok{15}\NormalTok{,}\DecValTok{1}\NormalTok{,}\SpecialCharTok{{-}}\DecValTok{900}\NormalTok{,}\SpecialCharTok{{-}}\DecValTok{450}\NormalTok{), }\AttributeTok{indices =} \FunctionTok{c}\NormalTok{(}\DecValTok{3}\NormalTok{,}\DecValTok{6}\NormalTok{,}\DecValTok{8}\NormalTok{,}\DecValTok{11}\NormalTok{,}\DecValTok{12}\NormalTok{))}
\FunctionTok{set.row}\NormalTok{(lprecdual, }\DecValTok{9}\NormalTok{, }\FunctionTok{c}\NormalTok{(}\DecValTok{1}\NormalTok{,}\DecValTok{12}\NormalTok{,}\DecValTok{1}\NormalTok{,}\SpecialCharTok{{-}}\DecValTok{900}\NormalTok{,}\SpecialCharTok{{-}}\DecValTok{450}\NormalTok{), }\AttributeTok{indices =} \FunctionTok{c}\NormalTok{(}\DecValTok{3}\NormalTok{,}\DecValTok{6}\NormalTok{,}\DecValTok{9}\NormalTok{,}\DecValTok{11}\NormalTok{,}\DecValTok{12}\NormalTok{))}
\end{Highlighting}
\end{Shaded}

\begin{Shaded}
\begin{Highlighting}[]
\CommentTok{\# Setting all the right hand side values}
\NormalTok{rhs2 }\OtherTok{\textless{}{-}} \FunctionTok{c}\NormalTok{(}\DecValTok{420}\NormalTok{,}\DecValTok{420}\NormalTok{,}\DecValTok{420}\NormalTok{,}\DecValTok{360}\NormalTok{,}\DecValTok{360}\NormalTok{,}\DecValTok{360}\NormalTok{,}\DecValTok{300}\NormalTok{,}\DecValTok{300}\NormalTok{,}\DecValTok{300}\NormalTok{)}
\FunctionTok{set.rhs}\NormalTok{(lprecdual, rhs2)}
\end{Highlighting}
\end{Shaded}

\begin{Shaded}
\begin{Highlighting}[]
\CommentTok{\# Now Setting the constraint type}
\FunctionTok{set.constr.type}\NormalTok{(lprecdual, }\FunctionTok{c}\NormalTok{(}\StringTok{"\textgreater{}="}\NormalTok{,}\StringTok{"\textgreater{}="}\NormalTok{,}\StringTok{"\textgreater{}="}\NormalTok{,}\StringTok{"\textgreater{}="}\NormalTok{,}\StringTok{"\textgreater{}="}\NormalTok{,}\StringTok{"\textgreater{}="}\NormalTok{,}\StringTok{"\textgreater{}="}\NormalTok{,}\StringTok{"\textgreater{}="}\NormalTok{,}\StringTok{"\textgreater{}="}\NormalTok{))}
\end{Highlighting}
\end{Shaded}

\begin{Shaded}
\begin{Highlighting}[]
\CommentTok{\# Setting up the boundary condiiton for the decision variables}
\FunctionTok{set.bounds}\NormalTok{(lprecdual, }\AttributeTok{lower =} \FunctionTok{c}\NormalTok{(}\DecValTok{0}\NormalTok{,}\DecValTok{0}\NormalTok{,}\DecValTok{0}\NormalTok{,}\DecValTok{0}\NormalTok{,}\DecValTok{0}\NormalTok{,}\DecValTok{0}\NormalTok{,}\DecValTok{0}\NormalTok{,}\DecValTok{0}\NormalTok{,}\DecValTok{0}\NormalTok{,}\SpecialCharTok{{-}}\FloatTok{1.00e+30}\NormalTok{,}\SpecialCharTok{{-}}\FloatTok{1.00e+30}\NormalTok{,}\SpecialCharTok{{-}}\FloatTok{1.00e+30}\NormalTok{))}
\end{Highlighting}
\end{Shaded}

\begin{Shaded}
\begin{Highlighting}[]
\CommentTok{\# Solving the linear program}
\FunctionTok{solve}\NormalTok{(lprecdual)}
\end{Highlighting}
\end{Shaded}

\begin{verbatim}
## [1] 0
\end{verbatim}

\begin{Shaded}
\begin{Highlighting}[]
\CommentTok{\# Reviewing the objective function value}
\FunctionTok{get.objective}\NormalTok{(lprecdual)}
\end{Highlighting}
\end{Shaded}

\begin{verbatim}
## [1] 0
\end{verbatim}

\begin{Shaded}
\begin{Highlighting}[]
\CommentTok{\# Getting the optimum decision variable values}
\FunctionTok{get.variables}\NormalTok{(lprecdual)}
\end{Highlighting}
\end{Shaded}

\begin{verbatim}
##  [1]  0.000000  0.000000  0.000000  0.000000  0.000000  0.000000  0.000000
##  [8]  0.000000  0.000000 -2.400000 -3.200000  5.733333
\end{verbatim}

\begin{Shaded}
\begin{Highlighting}[]
\FunctionTok{get.sensitivity.rhs}\NormalTok{(lprecdual)}
\end{Highlighting}
\end{Shaded}

\begin{verbatim}
## $duals
##  [1]     0     0     0     0     0     0     0     0     0   750   900   450
## [13] 13000 12000  5000   900  1200   750     0     0     0
## 
## $dualsfrom
##  [1]  4.200000e+02 -1.000000e+30 -1.000000e+30 -1.000000e+30  3.600000e+02
##  [6] -1.000000e+30 -1.000000e+30 -1.000000e+30  3.000000e+02 -1.000000e+30
## [11] -1.000000e+30 -1.000000e+30 -1.000000e+30 -2.382043e-10 -7.914723e-11
## [16] -6.331779e-10 -1.512103e-09 -1.511488e-09 -1.000000e+30 -1.000000e+30
## [21] -1.000000e+30
## 
## $dualstill
##  [1] 1.000000e+30 1.000000e+30 1.000000e+30 1.000000e+30 1.000000e+30
##  [6] 1.000000e+30 1.000000e+30 1.000000e+30 1.000000e+30 1.032000e+03
## [11] 7.199263e+02 1.152000e+03 1.889360e-10 4.799508e+01 9.600000e+01
## [16] 1.511488e-09 1.191021e-09 6.331779e-10 1.000000e+30 1.000000e+30
## [21] 1.000000e+30
\end{verbatim}

\begin{Shaded}
\begin{Highlighting}[]
\FunctionTok{get.sensitivity.obj}\NormalTok{(lprecdual)}
\end{Highlighting}
\end{Shaded}

\begin{verbatim}
## $objfrom
##  [1] 0 0 0 0 0 0 0 0 0 0 0 0
## 
## $objtill
##  [1] 1.00e+30 1.00e+30 1.00e+30 1.00e+30 1.00e+30 1.00e+30 1.00e+30 1.00e+30
##  [9] 1.00e+30 1.00e+30 1.00e+30 1.17e+05
\end{verbatim}

\end{document}
