% Options for packages loaded elsewhere
\PassOptionsToPackage{unicode}{hyperref}
\PassOptionsToPackage{hyphens}{url}
%
\documentclass[
]{article}
\title{Assignment\_2}
\author{Swetha}
\date{11/15/2022}

\usepackage{amsmath,amssymb}
\usepackage{lmodern}
\usepackage{iftex}
\ifPDFTeX
  \usepackage[T1]{fontenc}
  \usepackage[utf8]{inputenc}
  \usepackage{textcomp} % provide euro and other symbols
\else % if luatex or xetex
  \usepackage{unicode-math}
  \defaultfontfeatures{Scale=MatchLowercase}
  \defaultfontfeatures[\rmfamily]{Ligatures=TeX,Scale=1}
\fi
% Use upquote if available, for straight quotes in verbatim environments
\IfFileExists{upquote.sty}{\usepackage{upquote}}{}
\IfFileExists{microtype.sty}{% use microtype if available
  \usepackage[]{microtype}
  \UseMicrotypeSet[protrusion]{basicmath} % disable protrusion for tt fonts
}{}
\makeatletter
\@ifundefined{KOMAClassName}{% if non-KOMA class
  \IfFileExists{parskip.sty}{%
    \usepackage{parskip}
  }{% else
    \setlength{\parindent}{0pt}
    \setlength{\parskip}{6pt plus 2pt minus 1pt}}
}{% if KOMA class
  \KOMAoptions{parskip=half}}
\makeatother
\usepackage{xcolor}
\IfFileExists{xurl.sty}{\usepackage{xurl}}{} % add URL line breaks if available
\IfFileExists{bookmark.sty}{\usepackage{bookmark}}{\usepackage{hyperref}}
\hypersetup{
  pdftitle={Assignment\_2},
  pdfauthor={Swetha},
  hidelinks,
  pdfcreator={LaTeX via pandoc}}
\urlstyle{same} % disable monospaced font for URLs
\usepackage[margin=1in]{geometry}
\usepackage{color}
\usepackage{fancyvrb}
\newcommand{\VerbBar}{|}
\newcommand{\VERB}{\Verb[commandchars=\\\{\}]}
\DefineVerbatimEnvironment{Highlighting}{Verbatim}{commandchars=\\\{\}}
% Add ',fontsize=\small' for more characters per line
\usepackage{framed}
\definecolor{shadecolor}{RGB}{248,248,248}
\newenvironment{Shaded}{\begin{snugshade}}{\end{snugshade}}
\newcommand{\AlertTok}[1]{\textcolor[rgb]{0.94,0.16,0.16}{#1}}
\newcommand{\AnnotationTok}[1]{\textcolor[rgb]{0.56,0.35,0.01}{\textbf{\textit{#1}}}}
\newcommand{\AttributeTok}[1]{\textcolor[rgb]{0.77,0.63,0.00}{#1}}
\newcommand{\BaseNTok}[1]{\textcolor[rgb]{0.00,0.00,0.81}{#1}}
\newcommand{\BuiltInTok}[1]{#1}
\newcommand{\CharTok}[1]{\textcolor[rgb]{0.31,0.60,0.02}{#1}}
\newcommand{\CommentTok}[1]{\textcolor[rgb]{0.56,0.35,0.01}{\textit{#1}}}
\newcommand{\CommentVarTok}[1]{\textcolor[rgb]{0.56,0.35,0.01}{\textbf{\textit{#1}}}}
\newcommand{\ConstantTok}[1]{\textcolor[rgb]{0.00,0.00,0.00}{#1}}
\newcommand{\ControlFlowTok}[1]{\textcolor[rgb]{0.13,0.29,0.53}{\textbf{#1}}}
\newcommand{\DataTypeTok}[1]{\textcolor[rgb]{0.13,0.29,0.53}{#1}}
\newcommand{\DecValTok}[1]{\textcolor[rgb]{0.00,0.00,0.81}{#1}}
\newcommand{\DocumentationTok}[1]{\textcolor[rgb]{0.56,0.35,0.01}{\textbf{\textit{#1}}}}
\newcommand{\ErrorTok}[1]{\textcolor[rgb]{0.64,0.00,0.00}{\textbf{#1}}}
\newcommand{\ExtensionTok}[1]{#1}
\newcommand{\FloatTok}[1]{\textcolor[rgb]{0.00,0.00,0.81}{#1}}
\newcommand{\FunctionTok}[1]{\textcolor[rgb]{0.00,0.00,0.00}{#1}}
\newcommand{\ImportTok}[1]{#1}
\newcommand{\InformationTok}[1]{\textcolor[rgb]{0.56,0.35,0.01}{\textbf{\textit{#1}}}}
\newcommand{\KeywordTok}[1]{\textcolor[rgb]{0.13,0.29,0.53}{\textbf{#1}}}
\newcommand{\NormalTok}[1]{#1}
\newcommand{\OperatorTok}[1]{\textcolor[rgb]{0.81,0.36,0.00}{\textbf{#1}}}
\newcommand{\OtherTok}[1]{\textcolor[rgb]{0.56,0.35,0.01}{#1}}
\newcommand{\PreprocessorTok}[1]{\textcolor[rgb]{0.56,0.35,0.01}{\textit{#1}}}
\newcommand{\RegionMarkerTok}[1]{#1}
\newcommand{\SpecialCharTok}[1]{\textcolor[rgb]{0.00,0.00,0.00}{#1}}
\newcommand{\SpecialStringTok}[1]{\textcolor[rgb]{0.31,0.60,0.02}{#1}}
\newcommand{\StringTok}[1]{\textcolor[rgb]{0.31,0.60,0.02}{#1}}
\newcommand{\VariableTok}[1]{\textcolor[rgb]{0.00,0.00,0.00}{#1}}
\newcommand{\VerbatimStringTok}[1]{\textcolor[rgb]{0.31,0.60,0.02}{#1}}
\newcommand{\WarningTok}[1]{\textcolor[rgb]{0.56,0.35,0.01}{\textbf{\textit{#1}}}}
\usepackage{graphicx}
\makeatletter
\def\maxwidth{\ifdim\Gin@nat@width>\linewidth\linewidth\else\Gin@nat@width\fi}
\def\maxheight{\ifdim\Gin@nat@height>\textheight\textheight\else\Gin@nat@height\fi}
\makeatother
% Scale images if necessary, so that they will not overflow the page
% margins by default, and it is still possible to overwrite the defaults
% using explicit options in \includegraphics[width, height, ...]{}
\setkeys{Gin}{width=\maxwidth,height=\maxheight,keepaspectratio}
% Set default figure placement to htbp
\makeatletter
\def\fps@figure{htbp}
\makeatother
\setlength{\emergencystretch}{3em} % prevent overfull lines
\providecommand{\tightlist}{%
  \setlength{\itemsep}{0pt}\setlength{\parskip}{0pt}}
\setcounter{secnumdepth}{-\maxdimen} % remove section numbering
\ifLuaTeX
  \usepackage{selnolig}  % disable illegal ligatures
\fi

\begin{document}
\maketitle

\#\#Integer Programming

AP is a shipping service that guarantees overnight delivery of packages
in the continental US. The company has various hubs at major cities and
airports across the country. Packages are received at hubs, and then
shipped to intermediate hubs or to their final destination. The manager
of the AP hub in Cleveland is concerned about labor costs, and is
interested in determining the most effective way to schedule workers.
The hub operates seven days a week, and the number of packages it
handles varies from one day to another.

Setting default values to get a clean output

\begin{Shaded}
\begin{Highlighting}[]
\CommentTok{\#loading the .lp file}
\FunctionTok{getwd}\NormalTok{()}
\end{Highlighting}
\end{Shaded}

\begin{verbatim}
## [1] "C:/Users/mercy/OneDrive/Desktop/QMM/Assignment_6"
\end{verbatim}

\begin{Shaded}
\begin{Highlighting}[]
\FunctionTok{setwd}\NormalTok{(}\StringTok{"C:/Users/mercy/OneDrive/Desktop/QMM/Assignment\_6"}\NormalTok{)}
\end{Highlighting}
\end{Shaded}

\begin{Shaded}
\begin{Highlighting}[]
\CommentTok{\#Loading the lpSolveAPI Package}
\FunctionTok{library}\NormalTok{(}\StringTok{"lpSolveAPI"}\NormalTok{)}
\end{Highlighting}
\end{Shaded}

\begin{verbatim}
## Warning: package 'lpSolveAPI' was built under R version 4.1.3
\end{verbatim}

\begin{Shaded}
\begin{Highlighting}[]
\FunctionTok{library}\NormalTok{(tinytex)}

\CommentTok{\#Loading the lp file}
\NormalTok{AP\_HUB }\OtherTok{\textless{}{-}} \FunctionTok{read.lp}\NormalTok{(}\StringTok{"IntegerProgramming.lp"}\NormalTok{)}
\FunctionTok{print}\NormalTok{(AP\_HUB)}
\end{Highlighting}
\end{Shaded}

\begin{verbatim}
## Model name: 
##             x1   x2   x3   x4   x5   x6   x7        
## Minimize   775  800  800  800  800  775  750        
## Sunday       0    1    1    1    1    1    0  >=  18
## Monday       0    0    1    1    1    1    1  >=  27
## Tuesday      1    0    0    1    1    1    1  >=  22
## Wednesday    1    1    0    0    1    1    1  >=  26
## Thursday     1    1    1    0    0    1    1  >=  25
## Friday       1    1    1    1    0    0    1  >=  21
## Saturday     1    1    1    1    1    0    0  >=  19
## Kind       Std  Std  Std  Std  Std  Std  Std        
## Type       Int  Int  Int  Int  Int  Int  Int        
## Upper      Inf  Inf  Inf  Inf  Inf  Inf  Inf        
## Lower        0    0    0    0    0    0    0
\end{verbatim}

\begin{Shaded}
\begin{Highlighting}[]
\CommentTok{\#We are estimating the number of workers required for every day for week in the table below}
\NormalTok{Required\_Workers\_Daywise}\OtherTok{\textless{}{-}} \FunctionTok{matrix}\NormalTok{(}\FunctionTok{c}\NormalTok{(}\StringTok{"Sunday"}\NormalTok{,}\StringTok{"Monday"}\NormalTok{,}\StringTok{"Tuesday"}\NormalTok{,}\StringTok{"Wednesday"}\NormalTok{,}\StringTok{"Thursday"}\NormalTok{,}\StringTok{"Friday"}\NormalTok{,}\StringTok{"Saturday"}\NormalTok{,}
\DecValTok{18}\NormalTok{,}\DecValTok{27}\NormalTok{,}\DecValTok{22}\NormalTok{,}\DecValTok{26}\NormalTok{,}\DecValTok{25}\NormalTok{,}\DecValTok{21}\NormalTok{,}\DecValTok{19}\NormalTok{),}\AttributeTok{ncol=}\DecValTok{2}\NormalTok{,}\AttributeTok{byrow =}\NormalTok{ F)}
\FunctionTok{colnames}\NormalTok{(Required\_Workers\_Daywise) }\OtherTok{\textless{}{-}} \FunctionTok{c}\NormalTok{(}\StringTok{"No\_of\_Days\_per\_week"}\NormalTok{, }\StringTok{"No\_of\_Workers\_Required"}\NormalTok{)}
\FunctionTok{as.table}\NormalTok{(Required\_Workers\_Daywise)}
\end{Highlighting}
\end{Shaded}

\begin{verbatim}
##   No_of_Days_per_week No_of_Workers_Required
## A Sunday              18                    
## B Monday              27                    
## C Tuesday             22                    
## D Wednesday           26                    
## E Thursday            25                    
## F Friday              21                    
## G Saturday            19
\end{verbatim}

Package handlers at AP are guaranteed a five-day work week with two
consecutive days off. The base wage for the handlers is \$750 per week.
Workers working on Saturday or Sunday receive an additional \$25 per
day. The possible shifts and salaries for package handlers are

\begin{Shaded}
\begin{Highlighting}[]
\NormalTok{No\_of\_Day\_offs\_and\_wages }\OtherTok{\textless{}{-}} \FunctionTok{matrix}\NormalTok{(}\FunctionTok{c}\NormalTok{(}\DecValTok{1}\NormalTok{,}\DecValTok{2}\NormalTok{,}\DecValTok{3}\NormalTok{,}\DecValTok{4}\NormalTok{,}\DecValTok{5}\NormalTok{,}\DecValTok{6}\NormalTok{,}\DecValTok{7}\NormalTok{, }\StringTok{"Sunday and Monday"}\NormalTok{,}\StringTok{"Monday and Tuesday"}\NormalTok{,}\StringTok{"Tuesday and Wednesday"}\NormalTok{,}\StringTok{"Wednesday and Thursday"}\NormalTok{,}\StringTok{"Thursday and Friday"}\NormalTok{,}\StringTok{"Friday and Saturday"}\NormalTok{,}\StringTok{"Saturday and Sunday"}\NormalTok{,}\StringTok{"$775"}\NormalTok{,}\StringTok{"$800"}\NormalTok{,}\StringTok{"$800"}\NormalTok{,}\StringTok{"$800"}\NormalTok{,}\StringTok{"$800"}\NormalTok{,}\StringTok{"$775"}\NormalTok{,}\StringTok{"$750"}\NormalTok{),}\AttributeTok{ncol=}\DecValTok{3}\NormalTok{,}\AttributeTok{byrow=}\NormalTok{F)}
\FunctionTok{colnames}\NormalTok{(No\_of\_Day\_offs\_and\_wages) }\OtherTok{\textless{}{-}} \FunctionTok{c}\NormalTok{(}\StringTok{"Shifts"}\NormalTok{, }\StringTok{"Day\_Offs"}\NormalTok{, }\StringTok{"Wages"}\NormalTok{)}
\FunctionTok{as.table}\NormalTok{(No\_of\_Day\_offs\_and\_wages)}
\end{Highlighting}
\end{Shaded}

\begin{verbatim}
##   Shifts Day_Offs               Wages
## A 1      Sunday and Monday      $775 
## B 2      Monday and Tuesday     $800 
## C 3      Tuesday and Wednesday  $800 
## D 4      Wednesday and Thursday $800 
## E 5      Thursday and Friday    $800 
## F 6      Friday and Saturday    $775 
## G 7      Saturday and Sunday    $750
\end{verbatim}

\begin{Shaded}
\begin{Highlighting}[]
\CommentTok{\#Now Running the lp model}
\FunctionTok{solve}\NormalTok{(AP\_HUB)}
\end{Highlighting}
\end{Shaded}

\begin{verbatim}
## [1] 0
\end{verbatim}

By getting 0 as the value we get to know that there exists a model.

\begin{Shaded}
\begin{Highlighting}[]
\CommentTok{\#Objective Function}
\FunctionTok{get.objective}\NormalTok{(AP\_HUB)}
\end{Highlighting}
\end{Shaded}

\begin{verbatim}
## [1] 25675
\end{verbatim}

The overall cost to the company to ensure that total pay expenses are as
little as feasible and that there are enough workers available each day
to work is ``25,675\$''.

\begin{Shaded}
\begin{Highlighting}[]
\CommentTok{\#The number of workers whom is available to work each day}
\FunctionTok{get.variables}\NormalTok{(AP\_HUB)}
\end{Highlighting}
\end{Shaded}

\begin{verbatim}
## [1]  2  4  5  0  8  1 13
\end{verbatim}

The variables have values from x1, x2\ldots\ldots.x7 where, x1 = Number
of workers assigned to shift 1 = 2 x2 = Number of workers assigned to
shift 2 = 4 x3 = Number of workers assigned to shift 3 = 5 x4 = Number
of workers assigned to shift 4 = 0 x5 = Number of workers assigned to
shift 5 = 8 x6 = Number of workers assigned to shift 6 = 1 x7 = Number
of workers assigned to shift 7 = 13 With respect to the objective
function as well as the limits set by the business, we may determine how
many workers are available to work each day by the variable values
obtained. Sunday = x2 + x3 + x4 + x5 + x6 = 18 Workers Monday = x3 + x4
+ x5 + x6 + x7 = 27 Workers Tuesday = x4 + x5 + x6 + x7 + x1 = 24
Workers Wednesday = x5 + x6 + x7 + x1 + x2 = 28 Workers Thursday = x6 +
x7 + x1 + x2 + x3 = 25 Workers Friday = x7 + x1 + x2 + x3 + x4 = 24
Workers Saturday = x1 + x2 + x3 + x4 + x5 = 19 Workers

\end{document}
