% Options for packages loaded elsewhere
\PassOptionsToPackage{unicode}{hyperref}
\PassOptionsToPackage{hyphens}{url}
%
\documentclass[
]{article}
\title{Assignment\_3}
\author{Swetha}
\date{10/14/2022}

\usepackage{amsmath,amssymb}
\usepackage{lmodern}
\usepackage{iftex}
\ifPDFTeX
  \usepackage[T1]{fontenc}
  \usepackage[utf8]{inputenc}
  \usepackage{textcomp} % provide euro and other symbols
\else % if luatex or xetex
  \usepackage{unicode-math}
  \defaultfontfeatures{Scale=MatchLowercase}
  \defaultfontfeatures[\rmfamily]{Ligatures=TeX,Scale=1}
\fi
% Use upquote if available, for straight quotes in verbatim environments
\IfFileExists{upquote.sty}{\usepackage{upquote}}{}
\IfFileExists{microtype.sty}{% use microtype if available
  \usepackage[]{microtype}
  \UseMicrotypeSet[protrusion]{basicmath} % disable protrusion for tt fonts
}{}
\makeatletter
\@ifundefined{KOMAClassName}{% if non-KOMA class
  \IfFileExists{parskip.sty}{%
    \usepackage{parskip}
  }{% else
    \setlength{\parindent}{0pt}
    \setlength{\parskip}{6pt plus 2pt minus 1pt}}
}{% if KOMA class
  \KOMAoptions{parskip=half}}
\makeatother
\usepackage{xcolor}
\IfFileExists{xurl.sty}{\usepackage{xurl}}{} % add URL line breaks if available
\IfFileExists{bookmark.sty}{\usepackage{bookmark}}{\usepackage{hyperref}}
\hypersetup{
  pdftitle={Assignment\_3},
  pdfauthor={Swetha},
  hidelinks,
  pdfcreator={LaTeX via pandoc}}
\urlstyle{same} % disable monospaced font for URLs
\usepackage[margin=1in]{geometry}
\usepackage{color}
\usepackage{fancyvrb}
\newcommand{\VerbBar}{|}
\newcommand{\VERB}{\Verb[commandchars=\\\{\}]}
\DefineVerbatimEnvironment{Highlighting}{Verbatim}{commandchars=\\\{\}}
% Add ',fontsize=\small' for more characters per line
\usepackage{framed}
\definecolor{shadecolor}{RGB}{248,248,248}
\newenvironment{Shaded}{\begin{snugshade}}{\end{snugshade}}
\newcommand{\AlertTok}[1]{\textcolor[rgb]{0.94,0.16,0.16}{#1}}
\newcommand{\AnnotationTok}[1]{\textcolor[rgb]{0.56,0.35,0.01}{\textbf{\textit{#1}}}}
\newcommand{\AttributeTok}[1]{\textcolor[rgb]{0.77,0.63,0.00}{#1}}
\newcommand{\BaseNTok}[1]{\textcolor[rgb]{0.00,0.00,0.81}{#1}}
\newcommand{\BuiltInTok}[1]{#1}
\newcommand{\CharTok}[1]{\textcolor[rgb]{0.31,0.60,0.02}{#1}}
\newcommand{\CommentTok}[1]{\textcolor[rgb]{0.56,0.35,0.01}{\textit{#1}}}
\newcommand{\CommentVarTok}[1]{\textcolor[rgb]{0.56,0.35,0.01}{\textbf{\textit{#1}}}}
\newcommand{\ConstantTok}[1]{\textcolor[rgb]{0.00,0.00,0.00}{#1}}
\newcommand{\ControlFlowTok}[1]{\textcolor[rgb]{0.13,0.29,0.53}{\textbf{#1}}}
\newcommand{\DataTypeTok}[1]{\textcolor[rgb]{0.13,0.29,0.53}{#1}}
\newcommand{\DecValTok}[1]{\textcolor[rgb]{0.00,0.00,0.81}{#1}}
\newcommand{\DocumentationTok}[1]{\textcolor[rgb]{0.56,0.35,0.01}{\textbf{\textit{#1}}}}
\newcommand{\ErrorTok}[1]{\textcolor[rgb]{0.64,0.00,0.00}{\textbf{#1}}}
\newcommand{\ExtensionTok}[1]{#1}
\newcommand{\FloatTok}[1]{\textcolor[rgb]{0.00,0.00,0.81}{#1}}
\newcommand{\FunctionTok}[1]{\textcolor[rgb]{0.00,0.00,0.00}{#1}}
\newcommand{\ImportTok}[1]{#1}
\newcommand{\InformationTok}[1]{\textcolor[rgb]{0.56,0.35,0.01}{\textbf{\textit{#1}}}}
\newcommand{\KeywordTok}[1]{\textcolor[rgb]{0.13,0.29,0.53}{\textbf{#1}}}
\newcommand{\NormalTok}[1]{#1}
\newcommand{\OperatorTok}[1]{\textcolor[rgb]{0.81,0.36,0.00}{\textbf{#1}}}
\newcommand{\OtherTok}[1]{\textcolor[rgb]{0.56,0.35,0.01}{#1}}
\newcommand{\PreprocessorTok}[1]{\textcolor[rgb]{0.56,0.35,0.01}{\textit{#1}}}
\newcommand{\RegionMarkerTok}[1]{#1}
\newcommand{\SpecialCharTok}[1]{\textcolor[rgb]{0.00,0.00,0.00}{#1}}
\newcommand{\SpecialStringTok}[1]{\textcolor[rgb]{0.31,0.60,0.02}{#1}}
\newcommand{\StringTok}[1]{\textcolor[rgb]{0.31,0.60,0.02}{#1}}
\newcommand{\VariableTok}[1]{\textcolor[rgb]{0.00,0.00,0.00}{#1}}
\newcommand{\VerbatimStringTok}[1]{\textcolor[rgb]{0.31,0.60,0.02}{#1}}
\newcommand{\WarningTok}[1]{\textcolor[rgb]{0.56,0.35,0.01}{\textbf{\textit{#1}}}}
\usepackage{graphicx}
\makeatletter
\def\maxwidth{\ifdim\Gin@nat@width>\linewidth\linewidth\else\Gin@nat@width\fi}
\def\maxheight{\ifdim\Gin@nat@height>\textheight\textheight\else\Gin@nat@height\fi}
\makeatother
% Scale images if necessary, so that they will not overflow the page
% margins by default, and it is still possible to overwrite the defaults
% using explicit options in \includegraphics[width, height, ...]{}
\setkeys{Gin}{width=\maxwidth,height=\maxheight,keepaspectratio}
% Set default figure placement to htbp
\makeatletter
\def\fps@figure{htbp}
\makeatother
\setlength{\emergencystretch}{3em} % prevent overfull lines
\providecommand{\tightlist}{%
  \setlength{\itemsep}{0pt}\setlength{\parskip}{0pt}}
\setcounter{secnumdepth}{-\maxdimen} % remove section numbering
\ifLuaTeX
  \usepackage{selnolig}  % disable illegal ligatures
\fi

\begin{document}
\maketitle

Objective Function Minimize TC =
622x11+614x12+630x13+641x21+645x22+649x23 These are subject to following
constraints supply constraints: x11+x12+x13 \textgreater= 100
x21+x22+x23 \textgreater= 120 demand constraints: x11+x21 \textgreater=
80 x12+x22 \textgreater= 60 x13+x23 \textgreater= 70

Now allare subjected to non-negativity where xij\textgreater=0 where
i=1,2 and j= 1,2,3

\#loading packages

\begin{Shaded}
\begin{Highlighting}[]
\FunctionTok{library}\NormalTok{(Matrix)}
\FunctionTok{library}\NormalTok{(}\StringTok{"lpSolve"}\NormalTok{)}
\end{Highlighting}
\end{Shaded}

\begin{verbatim}
## Warning: package 'lpSolve' was built under R version 4.1.3
\end{verbatim}

\begin{Shaded}
\begin{Highlighting}[]
\NormalTok{display }\OtherTok{\textless{}{-}} \FunctionTok{matrix}\NormalTok{(}\FunctionTok{c}\NormalTok{(}\DecValTok{22}\NormalTok{,}\DecValTok{14}\NormalTok{,}\DecValTok{30}\NormalTok{,}\DecValTok{600}\NormalTok{,}\DecValTok{100}\NormalTok{,}
                  \DecValTok{16}\NormalTok{,}\DecValTok{20}\NormalTok{,}\DecValTok{24}\NormalTok{,}\DecValTok{625}\NormalTok{,}\DecValTok{120}\NormalTok{,}
                  \DecValTok{80}\NormalTok{,}\DecValTok{60}\NormalTok{,}\DecValTok{70}\NormalTok{,}\StringTok{"{-}"}\NormalTok{,}\StringTok{"210/220"}\NormalTok{),}\AttributeTok{ncol=}\DecValTok{5}\NormalTok{,}\AttributeTok{nrow=}\DecValTok{3}\NormalTok{,}\AttributeTok{byrow=}\ConstantTok{TRUE}\NormalTok{)}
 \FunctionTok{colnames}\NormalTok{(display) }\OtherTok{\textless{}{-}} \FunctionTok{c}\NormalTok{(}\StringTok{"Warehouse1"}\NormalTok{,}\StringTok{"Warehouse2"}\NormalTok{,}\StringTok{"Warehouse3"}\NormalTok{,}\StringTok{"Production Cost"}\NormalTok{,}\StringTok{"Production Capacity"}\NormalTok{)}
 \FunctionTok{rownames}\NormalTok{(display) }\OtherTok{\textless{}{-}} \FunctionTok{c}\NormalTok{(}\StringTok{"PlantA"}\NormalTok{,}\StringTok{"PlantB"}\NormalTok{,}\StringTok{"Monthly Demand"}\NormalTok{)}
\NormalTok{ display }\OtherTok{\textless{}{-}} \FunctionTok{as.table}\NormalTok{(display)}
\NormalTok{ display}
\end{Highlighting}
\end{Shaded}

\begin{verbatim}
##                Warehouse1 Warehouse2 Warehouse3 Production Cost
## PlantA         22         14         30         600            
## PlantB         16         20         24         625            
## Monthly Demand 80         60         70         -              
##                Production Capacity
## PlantA         100                
## PlantB         120                
## Monthly Demand 210/220
\end{verbatim}

\begin{Shaded}
\begin{Highlighting}[]
\NormalTok{  display1 }\OtherTok{\textless{}{-}} \FunctionTok{matrix}\NormalTok{(}\FunctionTok{c}\NormalTok{(}\DecValTok{622}\NormalTok{,}\DecValTok{614}\NormalTok{,}\DecValTok{630}\NormalTok{,}\DecValTok{0}\NormalTok{,}\DecValTok{100}\NormalTok{,}
                  \DecValTok{641}\NormalTok{,}\DecValTok{645}\NormalTok{,}\DecValTok{649}\NormalTok{,}\DecValTok{0}\NormalTok{,}\DecValTok{120}\NormalTok{,}
                  \DecValTok{80}\NormalTok{,}\DecValTok{60}\NormalTok{,}\DecValTok{70}\NormalTok{,}\DecValTok{10}\NormalTok{,}\DecValTok{220}\NormalTok{),}\AttributeTok{ncol=}\DecValTok{5}\NormalTok{,}\AttributeTok{nrow=}\DecValTok{3}\NormalTok{,}\AttributeTok{byrow=}\ConstantTok{TRUE}\NormalTok{)}
 \FunctionTok{colnames}\NormalTok{(display1) }\OtherTok{\textless{}{-}} \FunctionTok{c}\NormalTok{(}\StringTok{"Warehouse1"}\NormalTok{,}\StringTok{"Warehouse2"}\NormalTok{,}\StringTok{"Warehouse3"}\NormalTok{,}\StringTok{"Dummy"}\NormalTok{,}\StringTok{"Production Capacity"}\NormalTok{)}
 \FunctionTok{rownames}\NormalTok{(display1) }\OtherTok{\textless{}{-}} \FunctionTok{c}\NormalTok{(}\StringTok{"PlantA"}\NormalTok{,}\StringTok{"PlantB"}\NormalTok{,}\StringTok{"Monthly Demand"}\NormalTok{)}
\NormalTok{ display1 }\OtherTok{\textless{}{-}} \FunctionTok{as.table}\NormalTok{(display1)}
\NormalTok{ display1}
\end{Highlighting}
\end{Shaded}

\begin{verbatim}
##                Warehouse1 Warehouse2 Warehouse3 Dummy Production Capacity
## PlantA                622        614        630     0                 100
## PlantB                641        645        649     0                 120
## Monthly Demand         80         60         70    10                 220
\end{verbatim}

\#The balanced problem will be satisfied by this table. Below is a cost
totals matrix that we created.

\begin{Shaded}
\begin{Highlighting}[]
\NormalTok{totalcosts }\OtherTok{\textless{}{-}} \FunctionTok{matrix}\NormalTok{(}\FunctionTok{c}\NormalTok{(}\DecValTok{622}\NormalTok{,}\DecValTok{614}\NormalTok{,}\DecValTok{630}\NormalTok{,}\DecValTok{0}\NormalTok{,}
                   \DecValTok{641}\NormalTok{,}\DecValTok{645}\NormalTok{,}\DecValTok{649}\NormalTok{,}\DecValTok{0}\NormalTok{),}\AttributeTok{nrow=}\DecValTok{2}\NormalTok{, }\AttributeTok{byrow =} \ConstantTok{TRUE}\NormalTok{)}
\end{Highlighting}
\end{Shaded}

\#finding the production capacity in the matrix's row

\begin{Shaded}
\begin{Highlighting}[]
\NormalTok{row.rhs }\OtherTok{\textless{}{-}} \FunctionTok{c}\NormalTok{(}\DecValTok{100}\NormalTok{,}\DecValTok{120}\NormalTok{)}
\NormalTok{ row.signs }\OtherTok{\textless{}{-}} \FunctionTok{rep}\NormalTok{(}\StringTok{"\textless{}="}\NormalTok{, }\DecValTok{2}\NormalTok{)}
\end{Highlighting}
\end{Shaded}

\#using the double variable 10 at the end to determine the monthly
demand.

\begin{Shaded}
\begin{Highlighting}[]
\NormalTok{col.rhs }\OtherTok{\textless{}{-}} \FunctionTok{c}\NormalTok{(}\DecValTok{80}\NormalTok{,}\DecValTok{60}\NormalTok{,}\DecValTok{70}\NormalTok{,}\DecValTok{10}\NormalTok{)}
\NormalTok{ col.signs }\OtherTok{\textless{}{-}} \FunctionTok{rep}\NormalTok{(}\StringTok{"\textgreater{}="}\NormalTok{, }\DecValTok{4}\NormalTok{)}
\end{Highlighting}
\end{Shaded}

\#now we are ready to run LP Transport command

\begin{Shaded}
\begin{Highlighting}[]
\FunctionTok{lp.transport}\NormalTok{(totalcosts,}\StringTok{"min"}\NormalTok{,row.signs,row.rhs,col.signs,col.rhs)}
\end{Highlighting}
\end{Shaded}

\begin{verbatim}
## Success: the objective function is 132790
\end{verbatim}

\hypertarget{solution-matrix}{%
\section{solution matrix}\label{solution-matrix}}

\begin{Shaded}
\begin{Highlighting}[]
\FunctionTok{lp.transport}\NormalTok{(totalcosts, }\StringTok{"min"}\NormalTok{, row.signs, row.rhs, col.signs, col.rhs)}\SpecialCharTok{$}\NormalTok{solution}
\end{Highlighting}
\end{Shaded}

\begin{verbatim}
##      [,1] [,2] [,3] [,4]
## [1,]    0   60   40    0
## [2,]   80    0   30   10
\end{verbatim}

We can conclude from this that Z = \$132790. The results are as follows
for each of the variables: 60x12 which is the Warehouse 2 from plant A.
40x13 which is the Warehouse 3 from plant A. 80x21 which is the
Warehouse 1 from plant B. 30x23 which is the Warehouse 3 from plant B.
and because ``10'' shows up in the 4th Variable 10x24 is a ``throw away
variable''.

Question 2) We are aware that the number of constants in dual and the
number of variables in primal are the same. The primary of the LP is
asked in the first question. We shall maximize in the dual because we
choose to minimize in the primal. Let's utilize ``m'' and ``n'' as our
variables for the dual problem.

\begin{Shaded}
\begin{Highlighting}[]
\NormalTok{display2 }\OtherTok{\textless{}{-}} \FunctionTok{matrix}\NormalTok{(}\FunctionTok{c}\NormalTok{(}\DecValTok{622}\NormalTok{,}\DecValTok{614}\NormalTok{,}\DecValTok{630}\NormalTok{,}\DecValTok{100}\NormalTok{,}\StringTok{"m\_1"}\NormalTok{,}
                  \DecValTok{641}\NormalTok{,}\DecValTok{645}\NormalTok{,}\DecValTok{649}\NormalTok{,}\DecValTok{120}\NormalTok{,}\StringTok{"m\_2"}\NormalTok{,}
                  \DecValTok{80}\NormalTok{,}\DecValTok{60}\NormalTok{,}\DecValTok{70}\NormalTok{,}\DecValTok{220}\NormalTok{,}\StringTok{"{-}"}\NormalTok{,}
                  \StringTok{"n\_1"}\NormalTok{,}\StringTok{"n\_2"}\NormalTok{,}\StringTok{"n\_3"}\NormalTok{,}\StringTok{"{-}"}\NormalTok{,}\StringTok{"{-}"}\NormalTok{),}\AttributeTok{ncol=}\DecValTok{5}\NormalTok{,}\AttributeTok{nrow=}\DecValTok{4}\NormalTok{,}\AttributeTok{byrow=}\ConstantTok{TRUE}\NormalTok{)}
 \FunctionTok{colnames}\NormalTok{(display2) }\OtherTok{\textless{}{-}} \FunctionTok{c}\NormalTok{(}\StringTok{"W1"}\NormalTok{,}\StringTok{"W2"}\NormalTok{,}\StringTok{"W3"}\NormalTok{,}\StringTok{"Prod Cap"}\NormalTok{,}\StringTok{"Supply (Dual)"}\NormalTok{)}
 \FunctionTok{rownames}\NormalTok{(display2) }\OtherTok{\textless{}{-}} \FunctionTok{c}\NormalTok{(}\StringTok{"PlantA"}\NormalTok{,}\StringTok{"PlantB"}\NormalTok{,}\StringTok{"Monthly Demand"}\NormalTok{,}\StringTok{"Demand (Dual)"}\NormalTok{)}
\NormalTok{ display2 }\OtherTok{\textless{}{-}} \FunctionTok{as.table}\NormalTok{(display2)}
\NormalTok{ display2}
\end{Highlighting}
\end{Shaded}

\begin{verbatim}
##                W1  W2  W3  Prod Cap Supply (Dual)
## PlantA         622 614 630 100      m_1          
## PlantB         641 645 649 120      m_2          
## Monthly Demand 80  60  70  220      -            
## Demand (Dual)  n_1 n_2 n_3 -        -
\end{verbatim}

Now we are going to create our objective function based on the
constraints from the primal. Later we will use the objective function
from the primal to find the constants of the dual.

Maximize Z = 100m1+120m2+80n1+60n2+70n3

This objective function is subject to following constraints:

m1+n1 \textless= 622 m1+n2 \textless= 614 m1+n3 \textless= 630 m2+n1
\textless= 641 m2+n2 \textless= 645 m2+n3 \textless= 649

These variables were extracted from the linear programming function's
transposed primal matrix. Transposing the f.con into the matrix and
comparing it to the primal's above constants is an easy way to verify.
These are confined where m=1,2 \& n=1,2,3 and mk, nl

\#Constants of the primal are now the objective function variables.

\begin{Shaded}
\begin{Highlighting}[]
\NormalTok{ f.obj }\OtherTok{\textless{}{-}} \FunctionTok{c}\NormalTok{(}\DecValTok{100}\NormalTok{,}\DecValTok{120}\NormalTok{,}\DecValTok{80}\NormalTok{,}\DecValTok{60}\NormalTok{,}\DecValTok{70}\NormalTok{)}
 \CommentTok{\#transposed from the constraints matrix in the primal}
\NormalTok{ f.con }\OtherTok{\textless{}{-}} \FunctionTok{matrix}\NormalTok{(}\FunctionTok{c}\NormalTok{(}\DecValTok{1}\NormalTok{,}\DecValTok{0}\NormalTok{,}\DecValTok{1}\NormalTok{,}\DecValTok{0}\NormalTok{,}\DecValTok{0}\NormalTok{,}
                   \DecValTok{1}\NormalTok{,}\DecValTok{0}\NormalTok{,}\DecValTok{0}\NormalTok{,}\DecValTok{1}\NormalTok{,}\DecValTok{0}\NormalTok{,}
                   \DecValTok{1}\NormalTok{,}\DecValTok{0}\NormalTok{,}\DecValTok{0}\NormalTok{,}\DecValTok{0}\NormalTok{,}\DecValTok{1}\NormalTok{,}
                   \DecValTok{0}\NormalTok{,}\DecValTok{1}\NormalTok{,}\DecValTok{1}\NormalTok{,}\DecValTok{0}\NormalTok{,}\DecValTok{0}\NormalTok{,}
                   \DecValTok{0}\NormalTok{,}\DecValTok{1}\NormalTok{,}\DecValTok{0}\NormalTok{,}\DecValTok{1}\NormalTok{,}\DecValTok{0}\NormalTok{,}
                   \DecValTok{0}\NormalTok{,}\DecValTok{1}\NormalTok{,}\DecValTok{0}\NormalTok{,}\DecValTok{0}\NormalTok{,}\DecValTok{1}\NormalTok{),}\AttributeTok{nrow=}\DecValTok{6}\NormalTok{, }\AttributeTok{byrow =} \ConstantTok{TRUE}\NormalTok{)}
 \CommentTok{\#these change As we are MAX the dual not min}
\NormalTok{ f.dir }\OtherTok{\textless{}{-}} \FunctionTok{c}\NormalTok{(}\StringTok{"\textless{}="}\NormalTok{,}
 \StringTok{"\textless{}="}\NormalTok{,}
 \StringTok{"\textless{}="}\NormalTok{,}
 \StringTok{"\textless{}="}\NormalTok{,}
\StringTok{"\textless{}="}\NormalTok{, }\StringTok{"\textless{}="}\NormalTok{)}
\NormalTok{ f.rhs }\OtherTok{\textless{}{-}} \FunctionTok{c}\NormalTok{(}\DecValTok{622}\NormalTok{,}\DecValTok{614}\NormalTok{,}\DecValTok{630}\NormalTok{,}\DecValTok{641}\NormalTok{,}\DecValTok{645}\NormalTok{,}\DecValTok{649}\NormalTok{)}
 \FunctionTok{lp}\NormalTok{ (}\StringTok{"max"}\NormalTok{, f.obj, f.con, f.dir, f.rhs)}
\end{Highlighting}
\end{Shaded}

\begin{verbatim}
## Success: the objective function is 139120
\end{verbatim}

\begin{Shaded}
\begin{Highlighting}[]
\FunctionTok{lp}\NormalTok{ (}\StringTok{"max"}\NormalTok{, f.obj, f.con, f.dir, f.rhs)}\SpecialCharTok{$}\NormalTok{solution}
\end{Highlighting}
\end{Shaded}

\begin{verbatim}
## [1] 614 633   8   0  16
\end{verbatim}

So Z=139,120 dollars and variables are: m1 = 614 which represents plant
A m2 = 633 which represents Plant B n1 = 8 which represents Warehouse 1
n3 = 16 which represents Warehouse 3

OBSERVATION

The minimal Z=132790 (Primal) and the maximum Z=139120(Dual). What are
we trying to max/min in this problem. We found that we should not be
shipping from Plant(A/B) to all three Warehouses. We should be shipping
from:

60x12 which is 60 Units from Plant A to Warehouse 2. 40x13 which is 40
Units from Plant A to Warehouse 3. 80x21 which is 60 Units from Plant B
to Warehouse 1. 30x23 which is 60 Units from Plant B to Warehouse 3. Now
we want to Max the profits from each distribution in respect to
capacity.

Question 3)

m1 - n1 \textless= 622

then we subtract n1 to the other side to get m1 \textless= 622 - n1

To compute that value it would be \$614\textless=(-8+622) which is true.
We would continue to evaluate these equations:

m1 \textless= 622-n1===614\textless=622-8=614 = TRUE m1 \textless=
614-n2===614\textless=614-0=614 = TRUE m1 \textless=
630-n3===614\textless=630-16=614 = TRUE m2 \textless=
641-n1===633\textless=641-8=633 = TRUE m2 \textless=
645−n2===633\textless=645-0=645 = NOT TRUE m2 \textless=
649-n3===633\textless=649-16=633 = TRUE

\#By updating each column, we can test for the shadow price while also
learning from the Duality-and-Sensitivity. We change the 100 to 101 and
120 to 121 in our LP Transport.

\begin{Shaded}
\begin{Highlighting}[]
\NormalTok{row.rhs1 }\OtherTok{\textless{}{-}} \FunctionTok{c}\NormalTok{(}\DecValTok{101}\NormalTok{,}\DecValTok{120}\NormalTok{)}
\NormalTok{ row.signs1 }\OtherTok{\textless{}{-}} \FunctionTok{rep}\NormalTok{(}\StringTok{"\textless{}="}\NormalTok{, }\DecValTok{2}\NormalTok{)}
\NormalTok{ col.rhs1 }\OtherTok{\textless{}{-}} \FunctionTok{c}\NormalTok{(}\DecValTok{80}\NormalTok{,}\DecValTok{60}\NormalTok{,}\DecValTok{70}\NormalTok{,}\DecValTok{10}\NormalTok{)}
\NormalTok{ col.signs1 }\OtherTok{\textless{}{-}} \FunctionTok{rep}\NormalTok{(}\StringTok{"\textgreater{}="}\NormalTok{, }\DecValTok{4}\NormalTok{)}
\NormalTok{ row.rhs2 }\OtherTok{\textless{}{-}} \FunctionTok{c}\NormalTok{(}\DecValTok{100}\NormalTok{,}\DecValTok{121}\NormalTok{)}
\NormalTok{ row.signs2 }\OtherTok{\textless{}{-}} \FunctionTok{rep}\NormalTok{(}\StringTok{"\textless{}="}\NormalTok{, }\DecValTok{2}\NormalTok{)}
\NormalTok{ col.rhs2 }\OtherTok{\textless{}{-}} \FunctionTok{c}\NormalTok{(}\DecValTok{80}\NormalTok{,}\DecValTok{60}\NormalTok{,}\DecValTok{70}\NormalTok{,}\DecValTok{10}\NormalTok{)}
\NormalTok{ col.signs2 }\OtherTok{\textless{}{-}} \FunctionTok{rep}\NormalTok{(}\StringTok{"\textgreater{}="}\NormalTok{, }\DecValTok{4}\NormalTok{)}
 \FunctionTok{lp.transport}\NormalTok{(totalcosts,}\StringTok{"min"}\NormalTok{,row.signs,row.rhs,col.signs,col.rhs)}
\end{Highlighting}
\end{Shaded}

\begin{verbatim}
## Success: the objective function is 132790
\end{verbatim}

\begin{Shaded}
\begin{Highlighting}[]
 \FunctionTok{lp.transport}\NormalTok{(totalcosts,}\StringTok{"min"}\NormalTok{,row.signs1,row.rhs1,col.signs1,col.rhs1)}
\end{Highlighting}
\end{Shaded}

\begin{verbatim}
## Success: the objective function is 132771
\end{verbatim}

\begin{Shaded}
\begin{Highlighting}[]
\FunctionTok{lp.transport}\NormalTok{(totalcosts,}\StringTok{"min"}\NormalTok{,row.signs2,row.rhs2,col.signs2,col.rhs2)}
\end{Highlighting}
\end{Shaded}

\begin{verbatim}
## Success: the objective function is 132790
\end{verbatim}

Since we are taking the minimum of this particular function, the fact
that the number decreases by 19 indicates that the shadow price, which
was determined by adding 1 to each of the Plants and the primordial, is
19. Plant B, on the other hand, doesn't have a shadow price.
Additionally, we discovered that the dual variable n2 with the
relationship Marginal Revenue (MR) = Marginal Cost (MC). Considering the
equation that was It was determined that m2 = 645 -n2===633=645-0=645 =
NOT TRUE by utilizing m1-n1 = 622.

\begin{Shaded}
\begin{Highlighting}[]
\FunctionTok{lp}\NormalTok{ (}\StringTok{"max"}\NormalTok{, f.obj, f.con, f.dir, f.rhs)}\SpecialCharTok{$}\NormalTok{solution}
\end{Highlighting}
\end{Shaded}

\begin{verbatim}
## [1] 614 633   8   0  16
\end{verbatim}

n\_2 was = 0.

CONCLUSION: from the primal: 60x12 which is 60 Units from Plant A to
Warehouse 2. 40x13 which is 40 Units from Plant A to Warehouse 3. 80x21
which is 60 Units from Plant B to Warehouse 1. 30x23 which is 60 Units
from Plant B to Warehouse 3. from the dual We want the MR=MC. Five of
the six MR\textless=MC. The only equation that does not satisfy this
requirement is Plant B to Warehouse 2. We can see that from the primal
that we will not be shipping any AED device there.

\end{document}
